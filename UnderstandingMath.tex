\section{Understanding Math}
    \textit{In mathematics you don’t understand things, you just get used to them.}
    –John von Neumann

    This page is intended to raise your consciousness about the many ways there are
    to understand “understand” in math. A link to each chapter about understanding
    math is given here with some comments about the ideas expressed in it.

    \subsection{Definitions}
        \begin{itemize}
            \item{One piece of information you must have about a math concept is its \textbf{definition.}}

            \item{Every proof of a fact about the concept must be based on a
                \textbf{logical chain of reasoning starting with the definition.}}

            \item{The definition is one path to understanding the concept but not the only one.}
        \end{itemize}

    \subsection{Math objects}
        \begin{itemize}
            \item{\textbf{Mathematical} objects are what mathematics is “about”.}

            \item{The number 42 is a math object.}

            \item{The set of even positive integers is a math object.
                  Even though it is an infinite set, it is a \textit{single math object.}}

            \item{The function $f(x):={{\sin }}x$ is a math object. Its value can be
                  computed at many different numbers but it is a single, static
                  math object. You can visualize going along it from left to right,
                  which makes you go up and down over and over between 1 and -1, and it
                  is useful to visualize it that way (see Images and Metaphors), but it is a
                  single, static math object.}

            \item{We talk and think about math objects in some of the same ways
                  that we talk about physical objects, but in other ways they are
                  not like physical objects.}
        \end{itemize}

    \subsection{Math Structures}
        A \textbf{mathematical structure} is a special kind of math object defined as a set with some associated objects called structure.  Equivalence relations, partitions, groups and topological spaces are examples of mathematical structures.
